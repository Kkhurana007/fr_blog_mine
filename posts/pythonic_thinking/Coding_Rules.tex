% Options for packages loaded elsewhere
\PassOptionsToPackage{unicode}{hyperref}
\PassOptionsToPackage{hyphens}{url}
\PassOptionsToPackage{dvipsnames,svgnames,x11names}{xcolor}
%
\documentclass[
]{report}

\usepackage{amsmath,amssymb}
\usepackage{iftex}
\ifPDFTeX
  \usepackage[T1]{fontenc}
  \usepackage[utf8]{inputenc}
  \usepackage{textcomp} % provide euro and other symbols
\else % if luatex or xetex
  \usepackage{unicode-math}
  \defaultfontfeatures{Scale=MatchLowercase}
  \defaultfontfeatures[\rmfamily]{Ligatures=TeX,Scale=1}
\fi
\usepackage{lmodern}
\ifPDFTeX\else  
    % xetex/luatex font selection
\fi
% Use upquote if available, for straight quotes in verbatim environments
\IfFileExists{upquote.sty}{\usepackage{upquote}}{}
\IfFileExists{microtype.sty}{% use microtype if available
  \usepackage[]{microtype}
  \UseMicrotypeSet[protrusion]{basicmath} % disable protrusion for tt fonts
}{}
\makeatletter
\@ifundefined{KOMAClassName}{% if non-KOMA class
  \IfFileExists{parskip.sty}{%
    \usepackage{parskip}
  }{% else
    \setlength{\parindent}{0pt}
    \setlength{\parskip}{6pt plus 2pt minus 1pt}}
}{% if KOMA class
  \KOMAoptions{parskip=half}}
\makeatother
\usepackage{xcolor}
\setlength{\emergencystretch}{3em} % prevent overfull lines
\setcounter{secnumdepth}{-\maxdimen} % remove section numbering
% Make \paragraph and \subparagraph free-standing
\ifx\paragraph\undefined\else
  \let\oldparagraph\paragraph
  \renewcommand{\paragraph}[1]{\oldparagraph{#1}\mbox{}}
\fi
\ifx\subparagraph\undefined\else
  \let\oldsubparagraph\subparagraph
  \renewcommand{\subparagraph}[1]{\oldsubparagraph{#1}\mbox{}}
\fi

\usepackage{color}
\usepackage{fancyvrb}
\newcommand{\VerbBar}{|}
\newcommand{\VERB}{\Verb[commandchars=\\\{\}]}
\DefineVerbatimEnvironment{Highlighting}{Verbatim}{commandchars=\\\{\}}
% Add ',fontsize=\small' for more characters per line
\usepackage{framed}
\definecolor{shadecolor}{RGB}{241,243,245}
\newenvironment{Shaded}{\begin{snugshade}}{\end{snugshade}}
\newcommand{\AlertTok}[1]{\textcolor[rgb]{0.68,0.00,0.00}{#1}}
\newcommand{\AnnotationTok}[1]{\textcolor[rgb]{0.37,0.37,0.37}{#1}}
\newcommand{\AttributeTok}[1]{\textcolor[rgb]{0.40,0.45,0.13}{#1}}
\newcommand{\BaseNTok}[1]{\textcolor[rgb]{0.68,0.00,0.00}{#1}}
\newcommand{\BuiltInTok}[1]{\textcolor[rgb]{0.00,0.23,0.31}{#1}}
\newcommand{\CharTok}[1]{\textcolor[rgb]{0.13,0.47,0.30}{#1}}
\newcommand{\CommentTok}[1]{\textcolor[rgb]{0.37,0.37,0.37}{#1}}
\newcommand{\CommentVarTok}[1]{\textcolor[rgb]{0.37,0.37,0.37}{\textit{#1}}}
\newcommand{\ConstantTok}[1]{\textcolor[rgb]{0.56,0.35,0.01}{#1}}
\newcommand{\ControlFlowTok}[1]{\textcolor[rgb]{0.00,0.23,0.31}{#1}}
\newcommand{\DataTypeTok}[1]{\textcolor[rgb]{0.68,0.00,0.00}{#1}}
\newcommand{\DecValTok}[1]{\textcolor[rgb]{0.68,0.00,0.00}{#1}}
\newcommand{\DocumentationTok}[1]{\textcolor[rgb]{0.37,0.37,0.37}{\textit{#1}}}
\newcommand{\ErrorTok}[1]{\textcolor[rgb]{0.68,0.00,0.00}{#1}}
\newcommand{\ExtensionTok}[1]{\textcolor[rgb]{0.00,0.23,0.31}{#1}}
\newcommand{\FloatTok}[1]{\textcolor[rgb]{0.68,0.00,0.00}{#1}}
\newcommand{\FunctionTok}[1]{\textcolor[rgb]{0.28,0.35,0.67}{#1}}
\newcommand{\ImportTok}[1]{\textcolor[rgb]{0.00,0.46,0.62}{#1}}
\newcommand{\InformationTok}[1]{\textcolor[rgb]{0.37,0.37,0.37}{#1}}
\newcommand{\KeywordTok}[1]{\textcolor[rgb]{0.00,0.23,0.31}{#1}}
\newcommand{\NormalTok}[1]{\textcolor[rgb]{0.00,0.23,0.31}{#1}}
\newcommand{\OperatorTok}[1]{\textcolor[rgb]{0.37,0.37,0.37}{#1}}
\newcommand{\OtherTok}[1]{\textcolor[rgb]{0.00,0.23,0.31}{#1}}
\newcommand{\PreprocessorTok}[1]{\textcolor[rgb]{0.68,0.00,0.00}{#1}}
\newcommand{\RegionMarkerTok}[1]{\textcolor[rgb]{0.00,0.23,0.31}{#1}}
\newcommand{\SpecialCharTok}[1]{\textcolor[rgb]{0.37,0.37,0.37}{#1}}
\newcommand{\SpecialStringTok}[1]{\textcolor[rgb]{0.13,0.47,0.30}{#1}}
\newcommand{\StringTok}[1]{\textcolor[rgb]{0.13,0.47,0.30}{#1}}
\newcommand{\VariableTok}[1]{\textcolor[rgb]{0.07,0.07,0.07}{#1}}
\newcommand{\VerbatimStringTok}[1]{\textcolor[rgb]{0.13,0.47,0.30}{#1}}
\newcommand{\WarningTok}[1]{\textcolor[rgb]{0.37,0.37,0.37}{\textit{#1}}}

\providecommand{\tightlist}{%
  \setlength{\itemsep}{0pt}\setlength{\parskip}{0pt}}\usepackage{longtable,booktabs,array}
\usepackage{calc} % for calculating minipage widths
% Correct order of tables after \paragraph or \subparagraph
\usepackage{etoolbox}
\makeatletter
\patchcmd\longtable{\par}{\if@noskipsec\mbox{}\fi\par}{}{}
\makeatother
% Allow footnotes in longtable head/foot
\IfFileExists{footnotehyper.sty}{\usepackage{footnotehyper}}{\usepackage{footnote}}
\makesavenoteenv{longtable}
\usepackage{graphicx}
\makeatletter
\def\maxwidth{\ifdim\Gin@nat@width>\linewidth\linewidth\else\Gin@nat@width\fi}
\def\maxheight{\ifdim\Gin@nat@height>\textheight\textheight\else\Gin@nat@height\fi}
\makeatother
% Scale images if necessary, so that they will not overflow the page
% margins by default, and it is still possible to overwrite the defaults
% using explicit options in \includegraphics[width, height, ...]{}
\setkeys{Gin}{width=\maxwidth,height=\maxheight,keepaspectratio}
% Set default figure placement to htbp
\makeatletter
\def\fps@figure{htbp}
\makeatother

\makeatletter
\makeatother
\makeatletter
\makeatother
\makeatletter
\@ifpackageloaded{caption}{}{\usepackage{caption}}
\AtBeginDocument{%
\ifdefined\contentsname
  \renewcommand*\contentsname{Table of contents}
\else
  \newcommand\contentsname{Table of contents}
\fi
\ifdefined\listfigurename
  \renewcommand*\listfigurename{List of Figures}
\else
  \newcommand\listfigurename{List of Figures}
\fi
\ifdefined\listtablename
  \renewcommand*\listtablename{List of Tables}
\else
  \newcommand\listtablename{List of Tables}
\fi
\ifdefined\figurename
  \renewcommand*\figurename{Figure}
\else
  \newcommand\figurename{Figure}
\fi
\ifdefined\tablename
  \renewcommand*\tablename{Table}
\else
  \newcommand\tablename{Table}
\fi
}
\@ifpackageloaded{float}{}{\usepackage{float}}
\floatstyle{ruled}
\@ifundefined{c@chapter}{\newfloat{codelisting}{h}{lop}}{\newfloat{codelisting}{h}{lop}[chapter]}
\floatname{codelisting}{Listing}
\newcommand*\listoflistings{\listof{codelisting}{List of Listings}}
\makeatother
\makeatletter
\@ifpackageloaded{caption}{}{\usepackage{caption}}
\@ifpackageloaded{subcaption}{}{\usepackage{subcaption}}
\makeatother
\makeatletter
\@ifpackageloaded{tcolorbox}{}{\usepackage[skins,breakable]{tcolorbox}}
\makeatother
\makeatletter
\@ifundefined{shadecolor}{\definecolor{shadecolor}{rgb}{.97, .97, .97}}
\makeatother
\makeatletter
\makeatother
\makeatletter
\makeatother
\ifLuaTeX
  \usepackage{selnolig}  % disable illegal ligatures
\fi
\IfFileExists{bookmark.sty}{\usepackage{bookmark}}{\usepackage{hyperref}}
\IfFileExists{xurl.sty}{\usepackage{xurl}}{} % add URL line breaks if available
\urlstyle{same} % disable monospaced font for URLs
\hypersetup{
  pdftitle={Python Coding Rules},
  pdfauthor={Kunal Khurana},
  colorlinks=true,
  linkcolor={blue},
  filecolor={Maroon},
  citecolor={Blue},
  urlcolor={Blue},
  pdfcreator={LaTeX via pandoc}}

\title{Python Coding Rules}
\usepackage{etoolbox}
\makeatletter
\providecommand{\subtitle}[1]{% add subtitle to \maketitle
  \apptocmd{\@title}{\par {\large #1 \par}}{}{}
}
\makeatother
\subtitle{Python basics}
\author{Kunal Khurana}
\date{2023-12-06}

\begin{document}
\maketitle
\ifdefined\Shaded\renewenvironment{Shaded}{\begin{tcolorbox}[enhanced, interior hidden, sharp corners, boxrule=0pt, frame hidden, breakable, borderline west={3pt}{0pt}{shadecolor}]}{\end{tcolorbox}}\fi

\renewcommand*\contentsname{Table of contents}
{
\hypersetup{linkcolor=}
\setcounter{tocdepth}{2}
\tableofcontents
}
\hypertarget{general}{%
\section{General}\label{general}}

\hypertarget{c-style-formatting-strings-in-python-4-errors}{%
\subsubsection{C-style formatting strings in Python (4
errors)}\label{c-style-formatting-strings-in-python-4-errors}}

\begin{itemize}
\tightlist
\item
  reversing order gives traceback
\item
  difficult to read the code
\item
  using same value multiple times in tuple (repeat it in the right side)
\item
  dictionary formats
\end{itemize}

\hypertarget{write-helper-functions-instead-of-complex-expressions}{%
\subsubsection{Write helper functions instead of complex
expressions}\label{write-helper-functions-instead-of-complex-expressions}}

\begin{itemize}
\tightlist
\item
  Use if/else conditional to reduce visual noise
\item
  Moreover, if/else expression provides a more readable alternative over
  the boolean or/and in expressions.
\end{itemize}

\hypertarget{prefer-unpacking-over-indexing}{%
\subsection{Prefer Unpacking Over
Indexing}\label{prefer-unpacking-over-indexing}}

\begin{itemize}
\tightlist
\item
  use special syntax to unpack multiple values and keys in a single
  statement.
\end{itemize}

\hypertarget{prefer-enumerate-over-range}{%
\subsection{Prefer enumerate Over
range}\label{prefer-enumerate-over-range}}

\begin{itemize}
\tightlist
\item
  range (built-in funciton) is useful for loops
\item
  prefer enumerate instead of looping over a range
\end{itemize}

\begin{Shaded}
\begin{Highlighting}[]
\CommentTok{\# example of enumeration with list{-} }
\NormalTok{flavor\_list }\OperatorTok{=}\NormalTok{ [}\StringTok{\textquotesingle{}vanilla\textquotesingle{}}\NormalTok{, }\StringTok{\textquotesingle{}chocolate\textquotesingle{}}\NormalTok{, }\StringTok{\textquotesingle{}pecan\textquotesingle{}}\NormalTok{, }\StringTok{\textquotesingle{}strawberry\textquotesingle{}}\NormalTok{]}
\ControlFlowTok{for}\NormalTok{ flavor }\KeywordTok{in}\NormalTok{ flavor\_list:}
    \BuiltInTok{print}\NormalTok{(}\SpecialStringTok{f\textquotesingle{}}\SpecialCharTok{\{}\NormalTok{flavor}\SpecialCharTok{\}}\SpecialStringTok{ is delicious\textquotesingle{}}\NormalTok{)}
\end{Highlighting}
\end{Shaded}

\begin{verbatim}
vanilla is delicious
chocolate is delicious
pecan is delicious
strawberry is delicious
\end{verbatim}

\hypertarget{use-zip-to-process-iterators-in-parallel}{%
\subsection{Use zip to process Iterators in
parallel}\label{use-zip-to-process-iterators-in-parallel}}

\begin{Shaded}
\begin{Highlighting}[]
\NormalTok{names }\OperatorTok{=}\NormalTok{ [}\StringTok{\textquotesingle{}Kunal\textquotesingle{}}\NormalTok{, }\StringTok{\textquotesingle{}Xives\textquotesingle{}}\NormalTok{, }\StringTok{\textquotesingle{}pricila\textquotesingle{}}\NormalTok{]}
\NormalTok{counts }\OperatorTok{=}\NormalTok{ [}\BuiltInTok{len}\NormalTok{(n) }\ControlFlowTok{for}\NormalTok{ n }\KeywordTok{in}\NormalTok{ names]}
\BuiltInTok{print}\NormalTok{(counts)}
\end{Highlighting}
\end{Shaded}

\begin{verbatim}
[5, 5, 7]
\end{verbatim}

\begin{Shaded}
\begin{Highlighting}[]
\CommentTok{\# iterating over lenght of lists}
\NormalTok{longest\_name }\OperatorTok{=} \VariableTok{None}
\NormalTok{max\_count }\OperatorTok{=} \DecValTok{0}

\ControlFlowTok{for}\NormalTok{ i }\KeywordTok{in} \BuiltInTok{range}\NormalTok{(}\BuiltInTok{len}\NormalTok{(names)):}
\NormalTok{    count }\OperatorTok{=}\NormalTok{ counts[i]}
    \ControlFlowTok{if}\NormalTok{ count }\OperatorTok{\textgreater{}}\NormalTok{ max\_count:}
\NormalTok{        longest\_name }\OperatorTok{=}\NormalTok{ names[i]}
\NormalTok{        max\_count }\OperatorTok{=}\NormalTok{ count}
        
\BuiltInTok{print}\NormalTok{(longest\_name)}
\end{Highlighting}
\end{Shaded}

\begin{verbatim}
pricila
\end{verbatim}

\begin{Shaded}
\begin{Highlighting}[]
\CommentTok{\# we see that the above code is a bit noisy. }
\CommentTok{\# to imporve it, we\textquotesingle{}ll use the enumerate method}

\ControlFlowTok{for}\NormalTok{ i, name }\KeywordTok{in} \BuiltInTok{enumerate}\NormalTok{(names):}
\NormalTok{    count }\OperatorTok{=}\NormalTok{ counts[i]}
    \ControlFlowTok{if}\NormalTok{ count }\OperatorTok{\textgreater{}}\NormalTok{ max\_count:}
\NormalTok{        longest\_name }\OperatorTok{=}\NormalTok{ name}
\NormalTok{        max\_count }\OperatorTok{=}\NormalTok{ count}
\BuiltInTok{print}\NormalTok{(longest\_name)}
\end{Highlighting}
\end{Shaded}

\begin{verbatim}
pricila
\end{verbatim}

\begin{Shaded}
\begin{Highlighting}[]
\CommentTok{\# to improve it further, we\textquotesingle{}ll use the inbuilt zip function}

\ControlFlowTok{for}\NormalTok{ name, count }\KeywordTok{in} \BuiltInTok{zip}\NormalTok{(names, counts):}
    \ControlFlowTok{if}\NormalTok{ count }\OperatorTok{\textgreater{}}\NormalTok{ max\_count:}
\NormalTok{        longest\_name }\OperatorTok{=}\NormalTok{ name}
\NormalTok{        max\_count }\OperatorTok{=}\NormalTok{ count}

\BuiltInTok{print}\NormalTok{(longest\_name)}
\end{Highlighting}
\end{Shaded}

\begin{verbatim}
pricila
\end{verbatim}

\begin{Shaded}
\begin{Highlighting}[]
\CommentTok{\# zip\textquotesingle{}s behavior is different if counts are not updated}

\NormalTok{names.append(}\StringTok{\textquotesingle{}Rosy\textquotesingle{}}\NormalTok{)}
\ControlFlowTok{for}\NormalTok{ name, count }\KeywordTok{in} \BuiltInTok{zip}\NormalTok{(names, counts):}
    \BuiltInTok{print}\NormalTok{(name)}
\end{Highlighting}
\end{Shaded}

\begin{verbatim}
Kunal
Xives
pricila
\end{verbatim}

\begin{Shaded}
\begin{Highlighting}[]
\CommentTok{\# so, be careful when using iterators of different lenght. }

\CommentTok{\# consider using zip\_longest function from itertools instead}

\ImportTok{import}\NormalTok{ itertools}
\ControlFlowTok{for}\NormalTok{ name, count }\KeywordTok{in}\NormalTok{ itertools.zip\_longest (names, counts):}
    \BuiltInTok{print}\NormalTok{ (}\SpecialStringTok{f\textquotesingle{}}\SpecialCharTok{\{}\NormalTok{name}\SpecialCharTok{\}}\SpecialStringTok{: }\SpecialCharTok{\{}\NormalTok{count}\SpecialCharTok{\}}\SpecialStringTok{\textquotesingle{}}\NormalTok{)}
\end{Highlighting}
\end{Shaded}

\begin{verbatim}
Kunal: 5
Xives: 5
pricila: 7
Rosy: None
\end{verbatim}

\hypertarget{avoid-else-blocks-after-for-and-while-loops}{%
\subsection{Avoid `else' Blocks After `for' and `while'
Loops}\label{avoid-else-blocks-after-for-and-while-loops}}

\begin{Shaded}
\begin{Highlighting}[]
\CommentTok{\# for loops first}

\ControlFlowTok{for}\NormalTok{ i }\KeywordTok{in} \BuiltInTok{range}\NormalTok{(}\DecValTok{3}\NormalTok{):}
    \BuiltInTok{print}\NormalTok{(}\StringTok{\textquotesingle{}Loop\textquotesingle{}}\NormalTok{, i)}
\ControlFlowTok{else}\NormalTok{:}
    \BuiltInTok{print}\NormalTok{(}\StringTok{\textquotesingle{}Else block!\textquotesingle{}}\NormalTok{)}
\end{Highlighting}
\end{Shaded}

\begin{verbatim}
Loop 0
Loop 1
Loop 2
Else block!
\end{verbatim}

\begin{Shaded}
\begin{Highlighting}[]
\CommentTok{\# using break in the code}

\ControlFlowTok{for}\NormalTok{ i }\KeywordTok{in} \BuiltInTok{range}\NormalTok{(}\DecValTok{3}\NormalTok{):}
    \BuiltInTok{print}\NormalTok{(}\StringTok{\textquotesingle{}Loop\textquotesingle{}}\NormalTok{, i)}
    \ControlFlowTok{if}\NormalTok{ i }\OperatorTok{==} \DecValTok{1}\NormalTok{:}
        \ControlFlowTok{break}
        
\ControlFlowTok{else}\NormalTok{:}
    \BuiltInTok{print}\NormalTok{(}\StringTok{\textquotesingle{}Else block!\textquotesingle{}}\NormalTok{)}
\end{Highlighting}
\end{Shaded}

\begin{verbatim}
Loop 0
Loop 1
\end{verbatim}

\begin{Shaded}
\begin{Highlighting}[]
\CommentTok{\# else runs immediately if looped over an empty sequence}

\ControlFlowTok{for}\NormalTok{ x }\KeywordTok{in}\NormalTok{ []:}
    \BuiltInTok{print}\NormalTok{(}\StringTok{\textquotesingle{}Never runs\textquotesingle{}}\NormalTok{)}
\ControlFlowTok{else}\NormalTok{:}
    \BuiltInTok{print}\NormalTok{(}\StringTok{\textquotesingle{}For else block!\textquotesingle{}}\NormalTok{)}
\end{Highlighting}
\end{Shaded}

\begin{verbatim}
For else block!
\end{verbatim}

\begin{Shaded}
\begin{Highlighting}[]
\CommentTok{\# else also runs when while loops are initially false}
\ControlFlowTok{while} \VariableTok{False}\NormalTok{:}
    \BuiltInTok{print}\NormalTok{(}\StringTok{\textquotesingle{}Never runs\textquotesingle{}}\NormalTok{)}
\ControlFlowTok{else}\NormalTok{:}
    \BuiltInTok{print}\NormalTok{(}\StringTok{\textquotesingle{}While else block!\textquotesingle{}}\NormalTok{)}
\end{Highlighting}
\end{Shaded}

\begin{verbatim}
While else block!
\end{verbatim}

\begin{Shaded}
\begin{Highlighting}[]
\CommentTok{\#\# finding coprimes (having common divisor i.e. 1)}

\NormalTok{a }\OperatorTok{=} \DecValTok{11}
\NormalTok{b }\OperatorTok{=} \DecValTok{9}

\ControlFlowTok{for}\NormalTok{ i }\KeywordTok{in} \BuiltInTok{range}\NormalTok{(}\DecValTok{2}\NormalTok{, }\BuiltInTok{min}\NormalTok{(a, b) }\OperatorTok{+} \DecValTok{1}\NormalTok{):}
    \BuiltInTok{print}\NormalTok{ (}\StringTok{\textquotesingle{}Testing\textquotesingle{}}\NormalTok{, i)}
    \ControlFlowTok{if}\NormalTok{ a}\OperatorTok{\%}\NormalTok{ i }\OperatorTok{==} \DecValTok{0} \KeywordTok{and}\NormalTok{ b}\OperatorTok{\%}\NormalTok{i }\OperatorTok{==} \DecValTok{0}\NormalTok{:}
        \BuiltInTok{print}\NormalTok{(}\StringTok{\textquotesingle{}Not coprime\textquotesingle{}}\NormalTok{)}
        \ControlFlowTok{break}
\ControlFlowTok{else}\NormalTok{:}
    \BuiltInTok{print}\NormalTok{(}\StringTok{\textquotesingle{}coprime\textquotesingle{}}\NormalTok{)}
\end{Highlighting}
\end{Shaded}

\begin{verbatim}
Testing 2
Testing 3
Testing 4
Testing 5
Testing 6
Testing 7
Testing 8
Testing 9
coprime
\end{verbatim}

\hypertarget{prevent-repetition-with-assignment-expressions-such-as-warlus-operator}{%
\subsection{Prevent repetition with assignment Expressions such as
`warlus
operator'}\label{prevent-repetition-with-assignment-expressions-such-as-warlus-operator}}

\begin{Shaded}
\begin{Highlighting}[]
\CommentTok{\# Without the walrus operator}
\NormalTok{even\_numbers\_without\_walrus }\OperatorTok{=}\NormalTok{ []}
\NormalTok{count }\OperatorTok{=} \DecValTok{0}
\ControlFlowTok{while}\NormalTok{ count }\OperatorTok{\textless{}} \DecValTok{5}\NormalTok{:}
\NormalTok{    number }\OperatorTok{=}\NormalTok{ count }\OperatorTok{*} \DecValTok{2}
    \ControlFlowTok{if}\NormalTok{ number }\OperatorTok{\%} \DecValTok{2} \OperatorTok{==} \DecValTok{0}\NormalTok{:}
\NormalTok{        even\_numbers\_without\_walrus.append(number)}
\NormalTok{        count }\OperatorTok{+=} \DecValTok{1}

\BuiltInTok{print}\NormalTok{(even\_numbers\_without\_walrus)}
\end{Highlighting}
\end{Shaded}

\begin{verbatim}
[0, 2, 4, 6, 8]
\end{verbatim}

\begin{Shaded}
\begin{Highlighting}[]
\CommentTok{\# With the walrus operator}
\NormalTok{even\_numbers\_with\_walrus }\OperatorTok{=}\NormalTok{ []}
\NormalTok{count }\OperatorTok{=} \DecValTok{0}
\ControlFlowTok{while}\NormalTok{ count }\OperatorTok{\textless{}} \DecValTok{5}\NormalTok{:}
    \ControlFlowTok{if}\NormalTok{ (number }\OperatorTok{:=}\NormalTok{ count }\OperatorTok{*} \DecValTok{2}\NormalTok{) }\OperatorTok{\%} \DecValTok{2} \OperatorTok{==} \DecValTok{0}\NormalTok{:}
\NormalTok{        even\_numbers\_with\_walrus.append(number)}
\NormalTok{        count }\OperatorTok{+=} \DecValTok{1}

\BuiltInTok{print}\NormalTok{(even\_numbers\_with\_walrus)}
\end{Highlighting}
\end{Shaded}

\begin{verbatim}
[0, 2, 4, 6, 8]
\end{verbatim}

\hypertarget{lists-and-dictionaries}{%
\section{Lists and dictionaries}\label{lists-and-dictionaries}}

\hypertarget{know-how-to-slice-sequneces}{%
\subsection{Know how to slice
sequneces}\label{know-how-to-slice-sequneces}}

\begin{Shaded}
\begin{Highlighting}[]
\CommentTok{\#somelist [start:end]}
\NormalTok{a }\OperatorTok{=}\NormalTok{ [}\StringTok{\textquotesingle{}a\textquotesingle{}}\NormalTok{, }\StringTok{\textquotesingle{}b\textquotesingle{}}\NormalTok{, }\StringTok{\textquotesingle{}c\textquotesingle{}}\NormalTok{, }\StringTok{\textquotesingle{}d\textquotesingle{}}\NormalTok{, }\StringTok{\textquotesingle{}e\textquotesingle{}}\NormalTok{, }\StringTok{\textquotesingle{}f\textquotesingle{}}\NormalTok{]}
\BuiltInTok{print}\NormalTok{ (}\StringTok{\textquotesingle{}Middle two: \textquotesingle{}}\NormalTok{, a[}\DecValTok{2}\NormalTok{:}\DecValTok{4}\NormalTok{])}
\end{Highlighting}
\end{Shaded}

\begin{verbatim}
Middle two:  ['c', 'd']
\end{verbatim}

\hypertarget{avoid-striding-and-slicing-in-a-single-expression}{%
\subsection{Avoid striding and slicing in a single
expression}\label{avoid-striding-and-slicing-in-a-single-expression}}

\begin{Shaded}
\begin{Highlighting}[]
\NormalTok{b }\OperatorTok{=}\NormalTok{ [}\DecValTok{1}\NormalTok{, }\DecValTok{2}\NormalTok{, }\DecValTok{3}\NormalTok{, }\DecValTok{4}\NormalTok{, }\DecValTok{5}\NormalTok{, }\DecValTok{6}\NormalTok{]}
\NormalTok{odds }\OperatorTok{=}\NormalTok{ b[::}\DecValTok{2}\NormalTok{]}
\NormalTok{evens }\OperatorTok{=}\NormalTok{ b[}\DecValTok{1}\NormalTok{::}\DecValTok{2}\NormalTok{]}
\BuiltInTok{print}\NormalTok{(odds)}
\BuiltInTok{print}\NormalTok{(evens)}
\end{Highlighting}
\end{Shaded}

\begin{verbatim}
[1, 3, 5]
[2, 4, 6]
\end{verbatim}

\begin{Shaded}
\begin{Highlighting}[]
\CommentTok{\# stride syntax that can introduce bugs ; Avoid}
\NormalTok{c }\OperatorTok{=} \StringTok{b\textquotesingle{}rouge\textquotesingle{}}
\NormalTok{d }\OperatorTok{=}\NormalTok{ c[::}\OperatorTok{{-}}\DecValTok{1}\NormalTok{]}

\BuiltInTok{print}\NormalTok{(d)}
\end{Highlighting}
\end{Shaded}

\begin{verbatim}
b'eguor'
\end{verbatim}

\begin{Shaded}
\begin{Highlighting}[]
\NormalTok{x }\OperatorTok{=}\NormalTok{ [}\StringTok{\textquotesingle{}a\textquotesingle{}}\NormalTok{, }\StringTok{\textquotesingle{}b\textquotesingle{}}\NormalTok{, }\StringTok{\textquotesingle{}c\textquotesingle{}}\NormalTok{, }\StringTok{\textquotesingle{}d\textquotesingle{}}\NormalTok{, }\StringTok{\textquotesingle{}e\textquotesingle{}}\NormalTok{, }\StringTok{\textquotesingle{}f\textquotesingle{}}\NormalTok{, }\StringTok{\textquotesingle{}g\textquotesingle{}}\NormalTok{, }\StringTok{\textquotesingle{}h\textquotesingle{}}\NormalTok{]}
\BuiltInTok{print}\NormalTok{(x[}\DecValTok{2}\NormalTok{::}\DecValTok{2}\NormalTok{])     }\CommentTok{\# [\textquotesingle{}c\textquotesingle{}, \textquotesingle{}e\textquotesingle{}, \textquotesingle{}g\textquotesingle{}]}
\BuiltInTok{print}\NormalTok{(x[}\OperatorTok{{-}}\DecValTok{2}\NormalTok{::}\OperatorTok{{-}}\DecValTok{2}\NormalTok{])   }\CommentTok{\# [\textquotesingle{}g\textquotesingle{}, \textquotesingle{}e\textquotesingle{}, \textquotesingle{}c\textquotesingle{}, \textquotesingle{}a\textquotesingle{}]}
\BuiltInTok{print}\NormalTok{(x[}\OperatorTok{{-}}\DecValTok{2}\NormalTok{:}\DecValTok{2}\NormalTok{:}\OperatorTok{{-}}\DecValTok{2}\NormalTok{])  }\CommentTok{\# [\textquotesingle{}g\textquotesingle{}, \textquotesingle{}e\textquotesingle{}]  \#[start: stop : step]}
\BuiltInTok{print}\NormalTok{(x[}\DecValTok{2}\NormalTok{:}\DecValTok{2}\NormalTok{:}\OperatorTok{{-}}\DecValTok{2}\NormalTok{])   }\CommentTok{\# []}
\end{Highlighting}
\end{Shaded}

\begin{verbatim}
['c', 'e', 'g']
['g', 'e', 'c', 'a']
['g', 'e']
[]
\end{verbatim}

\hypertarget{perfect-catch-all-unpacking-over-slicing}{%
\subsection{Perfect Catch-`All Unpacking Over
Slicing'}\label{perfect-catch-all-unpacking-over-slicing}}

\begin{itemize}
\tightlist
\item
  Unpacking - extracting individual elements from a sequence (like a
  list or tuple) and assigning them to variables.
\item
  Slicing - selecting a subset of elements from a sequence.
\end{itemize}

\begin{Shaded}
\begin{Highlighting}[]
\CommentTok{\# Example sequence}
\NormalTok{numbers }\OperatorTok{=}\NormalTok{ [}\DecValTok{1}\NormalTok{, }\DecValTok{2}\NormalTok{, }\DecValTok{3}\NormalTok{, }\DecValTok{4}\NormalTok{, }\DecValTok{5}\NormalTok{, }\DecValTok{6}\NormalTok{, }\DecValTok{7}\NormalTok{, }\DecValTok{8}\NormalTok{, }\DecValTok{9}\NormalTok{, }\DecValTok{10}\NormalTok{]}

\CommentTok{\# Using slicing to get a portion of the sequence}
\NormalTok{subset }\OperatorTok{=}\NormalTok{ numbers[}\DecValTok{2}\NormalTok{:}\DecValTok{8}\NormalTok{]}

\CommentTok{\# Using unpacking to assign values to variables}
\NormalTok{first, }\OperatorTok{*}\NormalTok{middle, last }\OperatorTok{=}\NormalTok{ subset   }\CommentTok{\# *used for extended unpacking}

\CommentTok{\# Print the results}
\BuiltInTok{print}\NormalTok{(}\StringTok{"Subset:"}\NormalTok{, subset)}
\BuiltInTok{print}\NormalTok{(}\StringTok{"First element:"}\NormalTok{, first)}
\BuiltInTok{print}\NormalTok{(}\StringTok{"Middle elements:"}\NormalTok{, middle)}
\BuiltInTok{print}\NormalTok{(}\StringTok{"Last element:"}\NormalTok{, last)}
\end{Highlighting}
\end{Shaded}

\begin{verbatim}
Subset: [3, 4, 5, 6, 7, 8]
First element: 3
Middle elements: [4, 5, 6, 7]
Last element: 8
\end{verbatim}

\hypertarget{sort-by-complex-criteria-using-the-key-parameter}{%
\subsection{Sort by Complex Criteria using the `key'
parameter}\label{sort-by-complex-criteria-using-the-key-parameter}}

\begin{itemize}
\tightlist
\item
  sort method works for all built-in types (strings, floats, etc.), but
  it doesn't work for the classes, including a \textbf{repr} method for
  instance.
\end{itemize}

\begin{Shaded}
\begin{Highlighting}[]
\KeywordTok{class}\NormalTok{ Tool:}
    \KeywordTok{def} \FunctionTok{\_\_init\_\_}\NormalTok{(}\VariableTok{self}\NormalTok{, name, weight):}
        \VariableTok{self}\NormalTok{.name }\OperatorTok{=}\NormalTok{ name}
        \VariableTok{self}\NormalTok{.weight }\OperatorTok{=}\NormalTok{ weight}

    \KeywordTok{def} \FunctionTok{\_\_repr\_\_}\NormalTok{(}\VariableTok{self}\NormalTok{):}
        \ControlFlowTok{return} \SpecialStringTok{f\textquotesingle{}Tool(}\SpecialCharTok{\{}\VariableTok{self}\SpecialCharTok{.}\NormalTok{name}\SpecialCharTok{\}}\SpecialStringTok{, }\SpecialCharTok{\{}\VariableTok{self}\SpecialCharTok{.}\NormalTok{weight}\SpecialCharTok{\}}\SpecialStringTok{)\textquotesingle{}}

\CommentTok{\# Example usage of the Tool class}
\NormalTok{tools }\OperatorTok{=}\NormalTok{ [}
\NormalTok{    Tool(}\StringTok{\textquotesingle{}level\textquotesingle{}}\NormalTok{, }\FloatTok{3.5}\NormalTok{),}
\NormalTok{    Tool(}\StringTok{\textquotesingle{}hammer\textquotesingle{}}\NormalTok{, }\FloatTok{1.25}\NormalTok{),}
\NormalTok{    Tool(}\StringTok{\textquotesingle{}screwdriver\textquotesingle{}}\NormalTok{, }\FloatTok{0.5}\NormalTok{),}
\NormalTok{    Tool(}\StringTok{\textquotesingle{}chisel\textquotesingle{}}\NormalTok{, }\FloatTok{0.25}\NormalTok{),}
\NormalTok{]}

\CommentTok{\# tools.sort()   \#this will give us a traceback}
 
\CommentTok{\# Display the unsorted list of tools}
\BuiltInTok{print}\NormalTok{(}\StringTok{\textquotesingle{}Unsorted:\textquotesingle{}}\NormalTok{)}
\ControlFlowTok{for}\NormalTok{ tool }\KeywordTok{in}\NormalTok{ tools:}
    \BuiltInTok{print}\NormalTok{(}\BuiltInTok{repr}\NormalTok{(tool))}

\CommentTok{\# Sort the tools based on their names}
\NormalTok{tools.sort(key}\OperatorTok{=}\KeywordTok{lambda}\NormalTok{ x: x.name)}

\CommentTok{\# Display the sorted list of tools}
\BuiltInTok{print}\NormalTok{(}\StringTok{\textquotesingle{}}\CharTok{\textbackslash{}n}\StringTok{Sorted:\textquotesingle{}}\NormalTok{)}
\ControlFlowTok{for}\NormalTok{ tool }\KeywordTok{in}\NormalTok{ tools:}
    \BuiltInTok{print}\NormalTok{(tool)}
\end{Highlighting}
\end{Shaded}

\begin{verbatim}
Unsorted:
Tool(level, 3.5)
Tool(hammer, 1.25)
Tool(screwdriver, 0.5)
Tool(chisel, 0.25)

Sorted:
Tool(chisel, 0.25)
Tool(hammer, 1.25)
Tool(level, 3.5)
Tool(screwdriver, 0.5)
\end{verbatim}

\hypertarget{dictionaries-insertion-ordering-dict-types-dict-values}{%
\subsection{Dictionaries : insertion ordering, dict types, dict
values}\label{dictionaries-insertion-ordering-dict-types-dict-values}}

\begin{Shaded}
\begin{Highlighting}[]
\CommentTok{\# cutest baby animal}

\NormalTok{votes }\OperatorTok{=}\NormalTok{ \{}
    \StringTok{\textquotesingle{}otter\textquotesingle{}}\NormalTok{: }\DecValTok{1281}\NormalTok{,}
    \StringTok{\textquotesingle{}polar bear\textquotesingle{}}\NormalTok{: }\DecValTok{587}\NormalTok{,}
    \StringTok{\textquotesingle{}fox\textquotesingle{}}\NormalTok{: }\DecValTok{863}\NormalTok{,}
\NormalTok{\}}

\CommentTok{\# save the rank to an empty dictionary}
\KeywordTok{def}\NormalTok{ populate\_ranks(votes, ranks):  }\CommentTok{\#takes votes and ranks dictionary}
\NormalTok{    names }\OperatorTok{=} \BuiltInTok{list}\NormalTok{(votes.keys())}
\NormalTok{    names.sort(key}\OperatorTok{=}\NormalTok{votes.get, reverse}\OperatorTok{=}\VariableTok{True}\NormalTok{)}
    \ControlFlowTok{for}\NormalTok{ i, name }\KeywordTok{in} \BuiltInTok{enumerate}\NormalTok{(names, }\DecValTok{1}\NormalTok{):}
\NormalTok{        ranks[name] }\OperatorTok{=}\NormalTok{ i}
    
\CommentTok{\# function that returs the animal with hightest rank}
\KeywordTok{def}\NormalTok{ get\_winner(ranks):}
    \ControlFlowTok{return} \BuiltInTok{next}\NormalTok{(}\BuiltInTok{iter}\NormalTok{(ranks))}

\CommentTok{\# results}
\NormalTok{ranks }\OperatorTok{=}\NormalTok{ \{\}}
\NormalTok{populate\_ranks(votes, ranks)}
\BuiltInTok{print}\NormalTok{(ranks)}
\NormalTok{winner }\OperatorTok{=}\NormalTok{ get\_winner(ranks)}
\BuiltInTok{print}\NormalTok{(winner)}
\end{Highlighting}
\end{Shaded}

\begin{verbatim}
{'otter': 1, 'fox': 2, 'polar bear': 3}
otter
\end{verbatim}

\hypertarget{prefer-get-over-in-and-keyerror-to-handle-missing-dictionary-keys}{%
\subsection{Prefer `get' Over `in' and `KeyError' to handle missing
dictionary
keys}\label{prefer-get-over-in-and-keyerror-to-handle-missing-dictionary-keys}}

\begin{itemize}
\tightlist
\item
  accessing and assigning
\item
  for maintaining dictionaries, consider Counter class from the
  collections built-in module
\item
  setdefault is another shortened method other than get method, but
  readability is not clear. so, avoid it
\end{itemize}

\hypertarget{example-1}{%
\subsubsection{example 1}\label{example-1}}

\begin{Shaded}
\begin{Highlighting}[]
\NormalTok{bread }\OperatorTok{=}\NormalTok{ \{}
    \StringTok{\textquotesingle{}14grain\textquotesingle{}}\NormalTok{: }\DecValTok{4}\NormalTok{,}
    \StringTok{\textquotesingle{}multigrain\textquotesingle{}}\NormalTok{ : }\DecValTok{2}
\NormalTok{\}}

\CommentTok{\#1) \textquotesingle{}in\textquotesingle{} method}
\NormalTok{key }\OperatorTok{=} \StringTok{\textquotesingle{}wheat\textquotesingle{}}

\ControlFlowTok{if}\NormalTok{ key }\KeywordTok{in}\NormalTok{ bread:}
\NormalTok{    count }\OperatorTok{=}\NormalTok{ bread[key]}
\ControlFlowTok{else}\NormalTok{:}
\NormalTok{    count }\OperatorTok{=} \DecValTok{0}

\NormalTok{bread[key] }\OperatorTok{=}\NormalTok{ count }\OperatorTok{+} \DecValTok{1}  \CommentTok{\#incrementing the count by 1 for \textquotesingle{}wheat\textquotesingle{} key}
\end{Highlighting}
\end{Shaded}

\#2) `KeyError' method key = `wheat'

try: count = bread{[}key{]} except KeyError: count = 0

bread{[}key{]} = count + 1

\begin{Shaded}
\begin{Highlighting}[]
\CommentTok{\#3) \textquotesingle{}get\textquotesingle{} method {-} best one (shortest and clearest)}

\NormalTok{key }\OperatorTok{=} \StringTok{\textquotesingle{}oats\textquotesingle{}}

\NormalTok{count }\OperatorTok{=}\NormalTok{ bread.get(key,}\DecValTok{0}\NormalTok{)}
\NormalTok{bread[key] }\OperatorTok{=}\NormalTok{ count }\OperatorTok{+} \DecValTok{1}
\end{Highlighting}
\end{Shaded}

\begin{Shaded}
\begin{Highlighting}[]
\NormalTok{bread}
\end{Highlighting}
\end{Shaded}

\begin{verbatim}
{'14grain': 4, 'multigrain': 2, 'wheat': 2, 'oats': 1}
\end{verbatim}

\hypertarget{example-2}{%
\subsubsection{example 2}\label{example-2}}

\begin{Shaded}
\begin{Highlighting}[]
\CommentTok{\# more complex dictionary, to know who voted for which type of bread}

\NormalTok{votes }\OperatorTok{=}\NormalTok{ \{}
    \StringTok{\textquotesingle{}14grain\textquotesingle{}}\NormalTok{ : [}\StringTok{\textquotesingle{}Bob\textquotesingle{}}\NormalTok{, }\StringTok{\textquotesingle{}Ashley\textquotesingle{}}\NormalTok{, }\StringTok{\textquotesingle{}Suzan\textquotesingle{}}\NormalTok{, }\StringTok{\textquotesingle{}Susan\textquotesingle{}}\NormalTok{],}
    \StringTok{\textquotesingle{}multigrain\textquotesingle{}}\NormalTok{ : [}\StringTok{\textquotesingle{}Dikshita\textquotesingle{}}\NormalTok{, }\StringTok{\textquotesingle{}Kavya\textquotesingle{}}\NormalTok{],}
    \StringTok{\textquotesingle{}wheat\textquotesingle{}}\NormalTok{ : [}\StringTok{\textquotesingle{}Bhavna\textquotesingle{}}\NormalTok{, }\StringTok{\textquotesingle{}Shristi\textquotesingle{}}\NormalTok{],}
    \StringTok{\textquotesingle{}oats\textquotesingle{}}\NormalTok{ : [}\StringTok{\textquotesingle{}Nikumbh\textquotesingle{}}\NormalTok{]}
\NormalTok{\}}

\NormalTok{key }\OperatorTok{=} \StringTok{\textquotesingle{}kinoa\textquotesingle{}} 
\NormalTok{who }\OperatorTok{=} \StringTok{\textquotesingle{}Raph\textquotesingle{}}

\ControlFlowTok{if}\NormalTok{ key }\KeywordTok{in}\NormalTok{ votes: }
\NormalTok{    names }\OperatorTok{=}\NormalTok{ votes[key]}
\ControlFlowTok{else}\NormalTok{:}
\NormalTok{    votes[key] }\OperatorTok{=}\NormalTok{ names }\OperatorTok{=}\NormalTok{ []}
    
\NormalTok{names.append(who)}
\BuiltInTok{print}\NormalTok{ (votes)}
\end{Highlighting}
\end{Shaded}

\begin{verbatim}
{'14grain': ['Bob', 'Ashley', 'Suzan', 'Susan'], 'multigrain': ['Dikshita', 'Kavya'], 'wheat': ['Bhavna', 'Shristi'], 'oats': ['Nikumbh'], 'kinoa': ['Raph']}
\end{verbatim}

\begin{Shaded}
\begin{Highlighting}[]
\CommentTok{\# try except}

\ControlFlowTok{try}\NormalTok{:}
\NormalTok{    names }\OperatorTok{=}\NormalTok{ votes[key]}
\ControlFlowTok{except} \PreprocessorTok{KeyError}\NormalTok{:}
\NormalTok{    votes[key] }\OperatorTok{=}\NormalTok{ names }\OperatorTok{=}\NormalTok{[]}

\NormalTok{names.append(who)}
\BuiltInTok{print}\NormalTok{(votes)}
\end{Highlighting}
\end{Shaded}

\begin{verbatim}
{'14grain': ['Bob', 'Ashley', 'Suzan', 'Susan'], 'multigrain': ['Dikshita', 'Kavya'], 'wheat': ['Bhavna', 'Shristi'], 'oats': ['Nikumbh'], 'kinoa': ['Raph', 'Raph', 'Raph']}
\end{verbatim}

\begin{Shaded}
\begin{Highlighting}[]
\CommentTok{\# get method}
\NormalTok{names }\OperatorTok{=}\NormalTok{ votes.get(key)}
\ControlFlowTok{if}\NormalTok{ names }\KeywordTok{is} \VariableTok{None}\NormalTok{:}
\NormalTok{    votes[key] }\OperatorTok{=}\NormalTok{ names }\OperatorTok{=}\NormalTok{ []}
    
\NormalTok{names.append(who)}

\BuiltInTok{print}\NormalTok{(votes)}
\end{Highlighting}
\end{Shaded}

\begin{verbatim}
{'14grain': ['Bob', 'Ashley', 'Suzan', 'Susan'], 'multigrain': ['Dikshita', 'Kavya'], 'wheat': ['Bhavna', 'Shristi'], 'oats': ['Nikumbh', 'Raph', 'Raph', 'Raph', 'Raph'], 'kinoa': ['Raph', 'Raph', 'Raph']}
\end{verbatim}

\begin{Shaded}
\begin{Highlighting}[]
\CommentTok{\# prevent repetition}
\ControlFlowTok{if}\NormalTok{ (names }\OperatorTok{:=}\NormalTok{ votes.get(key)) }\KeywordTok{is} \VariableTok{None}\NormalTok{:}
\NormalTok{    votes[key] }\OperatorTok{=}\NormalTok{ names }\OperatorTok{=}\NormalTok{ []}
\NormalTok{names.append(who)}

\BuiltInTok{print}\NormalTok{(votes)}
\end{Highlighting}
\end{Shaded}

\begin{verbatim}
{'14grain': ['Bob', 'Ashley', 'Suzan', 'Susan'], 'multigrain': ['Dikshita', 'Kavya'], 'wheat': ['Bhavna', 'Shristi'], 'oats': ['Nikumbh', 'Raph', 'Raph', 'Raph', 'Raph', 'Raph', 'Raph'], 'kinoa': ['Raph', 'Raph', 'Raph']}
\end{verbatim}

\hypertarget{prefer-defaultdict-over-setdefault-to-handle-missing-items}{%
\subsection{Prefer `defaultdict' Over `Setdefault' to handle missing
items}\label{prefer-defaultdict-over-setdefault-to-handle-missing-items}}

\begin{Shaded}
\begin{Highlighting}[]
\CommentTok{\# list of countires and cities visited}
\NormalTok{visits }\OperatorTok{=}\NormalTok{ \{}
    \StringTok{\textquotesingle{}India\textquotesingle{}}\NormalTok{ : \{}\StringTok{\textquotesingle{}Punjab\textquotesingle{}}\NormalTok{, }\StringTok{\textquotesingle{}Rajastan\textquotesingle{}}\NormalTok{, }\StringTok{\textquotesingle{}Goa\textquotesingle{}}\NormalTok{, }\StringTok{\textquotesingle{}Himachal Pardesh\textquotesingle{}}\NormalTok{, }\StringTok{\textquotesingle{}Haryana\textquotesingle{}}\NormalTok{\},}
    \StringTok{\textquotesingle{}UAE\textquotesingle{}}\NormalTok{ : \{}\StringTok{\textquotesingle{}Dubai\textquotesingle{}}\NormalTok{\},}
    \StringTok{\textquotesingle{}Nepal\textquotesingle{}}\NormalTok{ : \{}\StringTok{\textquotesingle{}Kathmandu\textquotesingle{}}\NormalTok{\},}
    \StringTok{\textquotesingle{}Canada\textquotesingle{}}\NormalTok{ : \{}\StringTok{\textquotesingle{}Québec\textquotesingle{}}\NormalTok{, }\StringTok{\textquotesingle{}Ontario\textquotesingle{}}\NormalTok{\},}
\NormalTok{\}}


\CommentTok{\# using setdefalut method to add to the list (method 1)}

\NormalTok{visits.setdefault(}\StringTok{\textquotesingle{}France\textquotesingle{}}\NormalTok{, }\BuiltInTok{set}\NormalTok{()).add(}\StringTok{\textquotesingle{}Remi\textquotesingle{}}\NormalTok{)  }\CommentTok{\#short}

\ControlFlowTok{if}\NormalTok{ (japan }\OperatorTok{:=}\NormalTok{ visits.get(}\StringTok{\textquotesingle{}Japan\textquotesingle{}}\NormalTok{)) }\KeywordTok{is} \VariableTok{None}\NormalTok{:      }\CommentTok{\#long}
\NormalTok{    visits[}\StringTok{\textquotesingle{}Japan\textquotesingle{}}\NormalTok{] }\OperatorTok{=}\NormalTok{ japan }\OperatorTok{=} \BuiltInTok{set}\NormalTok{()}
\NormalTok{japan.add(}\StringTok{\textquotesingle{}Kyoto\textquotesingle{}}\NormalTok{)}

\BuiltInTok{print}\NormalTok{(visits)}
\end{Highlighting}
\end{Shaded}

\begin{verbatim}
{'India': {'Rajastan', 'Haryana', 'Punjab', 'Himachal Pardesh', 'Goa'}, 'UAE': {'Dubai'}, 'Nepal': {'Kathmandu'}, 'Canada': {'Ontario', 'Québec'}, 'France': {'Remi'}, 'Japan': {'Kyoto'}}
\end{verbatim}

\begin{Shaded}
\begin{Highlighting}[]
\CommentTok{\# how about i create a class then add places}

\ImportTok{from}\NormalTok{ collections }\ImportTok{import}\NormalTok{ defaultdict}

\KeywordTok{class}\NormalTok{ Visits:}
    \KeywordTok{def} \FunctionTok{\_\_init\_\_}\NormalTok{(}\VariableTok{self}\NormalTok{):}
       \VariableTok{self}\NormalTok{.data }\OperatorTok{=}\NormalTok{ defaultdict(}\BuiltInTok{set}\NormalTok{)}

    \KeywordTok{def}\NormalTok{ add(}\VariableTok{self}\NormalTok{, country, city):}
       \VariableTok{self}\NormalTok{.data[country].add(city)}

\NormalTok{visits }\OperatorTok{=}\NormalTok{ Visits()}
\NormalTok{visits.add(}\StringTok{\textquotesingle{}England\textquotesingle{}}\NormalTok{, }\StringTok{\textquotesingle{}Bath\textquotesingle{}}\NormalTok{)}
\NormalTok{visits.add(}\StringTok{\textquotesingle{}England\textquotesingle{}}\NormalTok{, }\StringTok{\textquotesingle{}London\textquotesingle{}}\NormalTok{)}
\BuiltInTok{print}\NormalTok{(visits.data)}
\end{Highlighting}
\end{Shaded}

\begin{verbatim}
defaultdict(<class 'set'>, {'England': {'Bath', 'London'}})
\end{verbatim}

\hypertarget{functions}{%
\section{Functions}\label{functions}}

\hypertarget{never-unpack-more-than-3-variables-when-fucntions-return-multiple-vaues}{%
\subsection{Never Unpack more than 3 variables when fucntions return
multiple
vaues}\label{never-unpack-more-than-3-variables-when-fucntions-return-multiple-vaues}}

\begin{Shaded}
\begin{Highlighting}[]
\CommentTok{\# Function returning multiple values}
\KeywordTok{def}\NormalTok{ get\_person\_details():}
\NormalTok{    name }\OperatorTok{=} \StringTok{"John"}
\NormalTok{    age }\OperatorTok{=} \DecValTok{30}
\NormalTok{    city }\OperatorTok{=} \StringTok{"Montréal"}
\NormalTok{    gender }\OperatorTok{=} \StringTok{"Male"}
    \ControlFlowTok{return}\NormalTok{ name, age, city, gender}

\CommentTok{\# Unpacking with three variables}
\NormalTok{name, age, city }\OperatorTok{=}\NormalTok{ get\_person\_details() }\CommentTok{\#return 3 }
\NormalTok{variables}

\CommentTok{\# Displaying the results}
\BuiltInTok{print}\NormalTok{(}\StringTok{"Name:"}\NormalTok{, name)}
\BuiltInTok{print}\NormalTok{(}\StringTok{"Age:"}\NormalTok{, age)}
\BuiltInTok{print}\NormalTok{(}\StringTok{"City:"}\NormalTok{, city)}
\end{Highlighting}
\end{Shaded}

\begin{verbatim}
ValueError: too many values to unpack (expected 3)
\end{verbatim}

\begin{Shaded}
\begin{Highlighting}[]
\CommentTok{\# Function returning multiple values}
\KeywordTok{def}\NormalTok{ get\_person\_details():}
\NormalTok{    name }\OperatorTok{=} \StringTok{"John"}
\NormalTok{    age }\OperatorTok{=} \DecValTok{30}
\NormalTok{    city }\OperatorTok{=} \StringTok{"Montréal"}
    \CommentTok{\#gender = "Male"}
    \ControlFlowTok{return}\NormalTok{ name, age, city}

\CommentTok{\# Unpacking with three variables}
\NormalTok{name, age, city }\OperatorTok{=}\NormalTok{ get\_person\_details() }\CommentTok{\#3 return variables}

\CommentTok{\# Displaying the results}
\BuiltInTok{print}\NormalTok{(}\StringTok{"Name:"}\NormalTok{, name)}
\BuiltInTok{print}\NormalTok{(}\StringTok{"Age:"}\NormalTok{, age)}
\BuiltInTok{print}\NormalTok{(}\StringTok{"City:"}\NormalTok{, city)}
\end{Highlighting}
\end{Shaded}

\begin{verbatim}
Name: John
Age: 30
City: Montréal
\end{verbatim}

\hypertarget{prefer-raising-exceptions-to-returning-none}{%
\subsection{Prefer raising exceptions to returning
None}\label{prefer-raising-exceptions-to-returning-none}}

\begin{Shaded}
\begin{Highlighting}[]
\CommentTok{\# Function that returns None on failure}
\KeywordTok{def}\NormalTok{ divide\_numbers(a, b):}
    \ControlFlowTok{if}\NormalTok{ b }\OperatorTok{==} \DecValTok{0}\NormalTok{:}
        \ControlFlowTok{return} \VariableTok{None}  \CommentTok{\# Indicating failure by returning None}
    \ControlFlowTok{else}\NormalTok{:}
        \ControlFlowTok{return}\NormalTok{ a }\OperatorTok{/}\NormalTok{ b}

\CommentTok{\# Using the function and checking for failure with None}
\NormalTok{result }\OperatorTok{=}\NormalTok{ divide\_numbers(}\DecValTok{10}\NormalTok{, }\DecValTok{2}\NormalTok{)}

\ControlFlowTok{if}\NormalTok{ result }\KeywordTok{is} \KeywordTok{not} \VariableTok{None}\NormalTok{:}
    \BuiltInTok{print}\NormalTok{(}\StringTok{"Result:"}\NormalTok{, result)}
\ControlFlowTok{else}\NormalTok{:}
    \BuiltInTok{print}\NormalTok{(}\StringTok{"Error: Cannot divide by zero."}\NormalTok{)}

\CommentTok{\# Using the function and checking for failure with None}
\NormalTok{result }\OperatorTok{=}\NormalTok{ divide\_numbers(}\DecValTok{10}\NormalTok{, }\DecValTok{0}\NormalTok{)}

\ControlFlowTok{if}\NormalTok{ result }\KeywordTok{is} \KeywordTok{not} \VariableTok{None}\NormalTok{:}
    \BuiltInTok{print}\NormalTok{(}\StringTok{"Result:"}\NormalTok{, result)}
\ControlFlowTok{else}\NormalTok{:}
    \BuiltInTok{print}\NormalTok{(}\StringTok{"Error: Cannot divide by zero."}\NormalTok{)}
\end{Highlighting}
\end{Shaded}

\begin{verbatim}
Result: 5.0
Error: Cannot divide by zero.
\end{verbatim}

\begin{Shaded}
\begin{Highlighting}[]
\CommentTok{\# Function that raises an exception on failure}
\KeywordTok{def}\NormalTok{ divide\_numbers(a, b):}
    \ControlFlowTok{if}\NormalTok{ b }\OperatorTok{==} \DecValTok{0}\NormalTok{:}
        \ControlFlowTok{raise} \PreprocessorTok{ValueError}\NormalTok{(}\StringTok{"Cannot divide by zero"}\NormalTok{)}
    \ControlFlowTok{else}\NormalTok{:}
        \ControlFlowTok{return}\NormalTok{ a }\OperatorTok{/}\NormalTok{ b}

\CommentTok{\# Using the function and handling the exception}
\ControlFlowTok{try}\NormalTok{:}
\NormalTok{    result }\OperatorTok{=}\NormalTok{ divide\_numbers(}\DecValTok{10}\NormalTok{, }\DecValTok{2}\NormalTok{)}
    \BuiltInTok{print}\NormalTok{(}\StringTok{"Result:"}\NormalTok{, result)}
\ControlFlowTok{except} \PreprocessorTok{ValueError} \ImportTok{as}\NormalTok{ e:}
    \BuiltInTok{print}\NormalTok{(}\StringTok{"Error:"}\NormalTok{, e)}

\CommentTok{\# Using the function and handling the exception}
\ControlFlowTok{try}\NormalTok{:}
\NormalTok{    result }\OperatorTok{=}\NormalTok{ divide\_numbers(}\DecValTok{10}\NormalTok{, }\DecValTok{0}\NormalTok{)}
    \BuiltInTok{print}\NormalTok{(}\StringTok{"Result:"}\NormalTok{, result)}
\ControlFlowTok{except} \PreprocessorTok{ValueError} \ImportTok{as}\NormalTok{ e:}
    \BuiltInTok{print}\NormalTok{(}\StringTok{"Error:"}\NormalTok{, e)}
\end{Highlighting}
\end{Shaded}

\begin{verbatim}
Result: 5.0
Error: Cannot divide by zero
\end{verbatim}

\hypertarget{know-how-closures-interact-with-variable-scope}{%
\subsection{Know how Closures Interact with Variable
Scope}\label{know-how-closures-interact-with-variable-scope}}

\begin{itemize}
\tightlist
\item
  It is better to write a helper class compared to non-local or helper
  function.
\item
  used specifically when we want to priortise certain groups in a
  function.
\end{itemize}

\begin{Shaded}
\begin{Highlighting}[]
\KeywordTok{class}\NormalTok{ Sorter:}
    \KeywordTok{def} \FunctionTok{\_\_init\_\_}\NormalTok{(}\VariableTok{self}\NormalTok{, group):}
        \VariableTok{self}\NormalTok{.group }\OperatorTok{=}\NormalTok{ group}
        \VariableTok{self}\NormalTok{.found }\OperatorTok{=} \VariableTok{False}

    \KeywordTok{def} \FunctionTok{\_\_call\_\_}\NormalTok{(}\VariableTok{self}\NormalTok{, x):}
        \ControlFlowTok{if}\NormalTok{ x }\KeywordTok{in} \VariableTok{self}\NormalTok{.group:}
            \VariableTok{self}\NormalTok{.found }\OperatorTok{=} \VariableTok{True}
            \ControlFlowTok{return}\NormalTok{ (}\DecValTok{0}\NormalTok{, x)}
        \ControlFlowTok{else}\NormalTok{:}
            \ControlFlowTok{return}\NormalTok{ (}\DecValTok{1}\NormalTok{, x)}

\CommentTok{\# Example usage}
\NormalTok{group }\OperatorTok{=}\NormalTok{ \{}\DecValTok{2}\NormalTok{, }\DecValTok{4}\NormalTok{, }\DecValTok{6}\NormalTok{\}}
\NormalTok{numbers }\OperatorTok{=}\NormalTok{ [}\DecValTok{5}\NormalTok{, }\DecValTok{3}\NormalTok{, }\DecValTok{2}\NormalTok{, }\DecValTok{1}\NormalTok{, }\DecValTok{4}\NormalTok{]}

\NormalTok{sorter }\OperatorTok{=}\NormalTok{ Sorter(group)}
\NormalTok{numbers.sort(key}\OperatorTok{=}\NormalTok{sorter)}

\CommentTok{\# Display the sorted list}
\BuiltInTok{print}\NormalTok{(}\StringTok{"Sorted List:"}\NormalTok{, numbers)}

\CommentTok{\# Check if any item from the group is found during sorting}
\ControlFlowTok{assert}\NormalTok{ sorter.found }\KeywordTok{is} \VariableTok{True}
\end{Highlighting}
\end{Shaded}

\begin{verbatim}
Sorted List: [2, 4, 1, 3, 5]
\end{verbatim}

\hypertarget{reduce-visual-noise-with-variable-positional-arguments}{%
\subsection{Reduce Visual Noise with Variable Positional
Arguments}\label{reduce-visual-noise-with-variable-positional-arguments}}

*args is not suggested for two reasons-

\begin{verbatim}
1) Optional positional arguments are always turned into a tuple before they are passed to a function. Uses a lot of memory and could crash a function.

2) Doesn't provide value inclusive of the new argument. Hence, no use of adding an additional argument. 
\end{verbatim}

\begin{Shaded}
\begin{Highlighting}[]
\CommentTok{\# Original function with *args}
\KeywordTok{def}\NormalTok{ example\_function(}\OperatorTok{*}\NormalTok{args):}
    \CommentTok{\# Existing functionality using args}
\NormalTok{    total }\OperatorTok{=} \BuiltInTok{sum}\NormalTok{(args)}
    \ControlFlowTok{return}\NormalTok{ total}

\CommentTok{\# Example usage}
\NormalTok{result }\OperatorTok{=}\NormalTok{ example\_function(}\DecValTok{1}\NormalTok{, }\DecValTok{2}\NormalTok{, }\DecValTok{3}\NormalTok{)}
\BuiltInTok{print}\NormalTok{(}\StringTok{"Result:"}\NormalTok{, result)}

\CommentTok{\# Attempt to add a new positional argument}
\CommentTok{\# This would break existing callers}
\KeywordTok{def}\NormalTok{ updated\_function(new\_arg, }\OperatorTok{*}\NormalTok{args):}
\NormalTok{     total }\OperatorTok{=} \BuiltInTok{sum}\NormalTok{(args) }\OperatorTok{+}\NormalTok{ new\_arg}
     \ControlFlowTok{return}\NormalTok{ total}

\NormalTok{result2 }\OperatorTok{=}\NormalTok{ updated\_function(}\DecValTok{4}\NormalTok{,}\DecValTok{5}\NormalTok{)}
\BuiltInTok{print}\NormalTok{(}\StringTok{\textquotesingle{}Result2:\textquotesingle{}}\NormalTok{, result2)}
\end{Highlighting}
\end{Shaded}

\begin{verbatim}
Result: 6
Result2: 9
\end{verbatim}

\hypertarget{provide-optional-behavior-with-keyword-arguments}{%
\subsection{Provide Optional Behavior with Keyword
Arguments}\label{provide-optional-behavior-with-keyword-arguments}}

\begin{Shaded}
\begin{Highlighting}[]
\KeywordTok{def}\NormalTok{ calculate\_rectangle\_area(length, width):}
    \ControlFlowTok{return}\NormalTok{ length }\OperatorTok{*}\NormalTok{ width}
\end{Highlighting}
\end{Shaded}

\begin{Shaded}
\begin{Highlighting}[]
\KeywordTok{def}\NormalTok{ calculate\_rectangle\_area(length, width}\OperatorTok{=}\VariableTok{None}\NormalTok{):}
    \ControlFlowTok{if}\NormalTok{ width }\KeywordTok{is} \KeywordTok{not} \VariableTok{None}\NormalTok{:}
        \ControlFlowTok{return}\NormalTok{ length }\OperatorTok{*}\NormalTok{ width}
    \ControlFlowTok{else}\NormalTok{:}
        \CommentTok{\# If width is not provided, assume it\textquotesingle{}s a square (width = length)}
        \ControlFlowTok{return}\NormalTok{ length }\OperatorTok{*}\NormalTok{ length}
\end{Highlighting}
\end{Shaded}

\begin{Shaded}
\begin{Highlighting}[]
\NormalTok{area1 }\OperatorTok{=}\NormalTok{ calculate\_rectangle\_area(}\DecValTok{5}\NormalTok{, }\DecValTok{3}\NormalTok{)  }\CommentTok{\# Calculates area of a rectangle}
\NormalTok{area2 }\OperatorTok{=}\NormalTok{ calculate\_rectangle\_area(}\DecValTok{4}\NormalTok{)     }\CommentTok{\# Assumes it\textquotesingle{}s a square with side length 4}

\BuiltInTok{print}\NormalTok{(area1)}
\BuiltInTok{print}\NormalTok{(area2)}
\end{Highlighting}
\end{Shaded}

\begin{verbatim}
15
16
\end{verbatim}

\hypertarget{use-none-and-docstrings-to-specify-dynamic-default-arguments}{%
\subsection{Use None and Docstrings to Specify Dynamic Default
Arguments}\label{use-none-and-docstrings-to-specify-dynamic-default-arguments}}

\begin{Shaded}
\begin{Highlighting}[]
\ImportTok{from}\NormalTok{ datetime }\ImportTok{import}\NormalTok{ datetime}

\KeywordTok{def}\NormalTok{ log\_message(message, timestamp}\OperatorTok{=}\VariableTok{None}\NormalTok{):}
    \CommentTok{"""}
\CommentTok{    Log a message with an optional timestamp.}

\CommentTok{    Parameters:}
\CommentTok{    {-} message (str): The message to be logged.}
\CommentTok{    {-} timestamp (datetime, optional): The timestamp for the log message.}
\CommentTok{      Defaults to the current time if not provided.}
\CommentTok{    """}
    \ControlFlowTok{if}\NormalTok{ timestamp }\KeywordTok{is} \VariableTok{None}\NormalTok{:}
\NormalTok{        timestamp }\OperatorTok{=}\NormalTok{ datetime.now()}

    \BuiltInTok{print}\NormalTok{(}\SpecialStringTok{f"}\SpecialCharTok{\{}\NormalTok{timestamp}\SpecialCharTok{\}}\SpecialStringTok{: }\SpecialCharTok{\{}\NormalTok{message}\SpecialCharTok{\}}\SpecialStringTok{"}\NormalTok{)}

\CommentTok{\# Example usage}
\NormalTok{log\_message(}\StringTok{"Error occurred"}\NormalTok{)  }\CommentTok{\# Logs the message with the current timestamp}
\NormalTok{log\_message(}\StringTok{"Warning"}\NormalTok{, timestamp}\OperatorTok{=}\NormalTok{datetime(}\DecValTok{2023}\NormalTok{, }\DecValTok{1}\NormalTok{, }\DecValTok{1}\NormalTok{))  }\CommentTok{\# Logs the message with a specific timestamp}
\end{Highlighting}
\end{Shaded}

\begin{verbatim}
2023-12-05 18:49:25.657917: Error occurred
2023-01-01 00:00:00: Warning
\end{verbatim}

\hypertarget{define-function-decorators-with-funtools.wraps}{%
\subsection{Define Function Decorators with
funtools.wraps}\label{define-function-decorators-with-funtools.wraps}}

\begin{itemize}
\tightlist
\item
  Decorator in Python is a function that takes another function as input
  and extends or modifies the behavior of the latter function.
\item
  In this case, the trace decorator is designed to print information
  about the function calls.
\end{itemize}

\begin{Shaded}
\begin{Highlighting}[]
\KeywordTok{def}\NormalTok{ trace(func):}
    \KeywordTok{def}\NormalTok{ wrapper(}\OperatorTok{*}\NormalTok{args, }\OperatorTok{**}\NormalTok{kwargs):}
\NormalTok{        result }\OperatorTok{=}\NormalTok{ func(}\OperatorTok{*}\NormalTok{args, }\OperatorTok{**}\NormalTok{kwargs)}
        \BuiltInTok{print}\NormalTok{(}\SpecialStringTok{f\textquotesingle{}}\SpecialCharTok{\{}\NormalTok{func}\SpecialCharTok{.}\VariableTok{\_\_name\_\_}\SpecialCharTok{\}}\SpecialStringTok{(}\SpecialCharTok{\{}\NormalTok{args}\SpecialCharTok{!r\}}\SpecialStringTok{, }\SpecialCharTok{\{}\NormalTok{kwargs}\SpecialCharTok{!r\}}\SpecialStringTok{) \textquotesingle{}}
              \SpecialStringTok{f\textquotesingle{}{-}\textgreater{} }\SpecialCharTok{\{}\NormalTok{result}\SpecialCharTok{!r\}}\SpecialStringTok{\textquotesingle{}}\NormalTok{)}
        \ControlFlowTok{return}\NormalTok{ result}
    \ControlFlowTok{return}\NormalTok{ wrapper}
\end{Highlighting}
\end{Shaded}

\begin{Shaded}
\begin{Highlighting}[]
\AttributeTok{@trace}
\KeywordTok{def}\NormalTok{ example\_function(x, y):}
    \ControlFlowTok{return}\NormalTok{ x }\OperatorTok{*}\NormalTok{ y}

\NormalTok{result }\OperatorTok{=}\NormalTok{ example\_function(}\DecValTok{3}\NormalTok{, }\DecValTok{4}\NormalTok{)}
\end{Highlighting}
\end{Shaded}

\begin{verbatim}
example_function((3, 4), {}) -> 12
\end{verbatim}

\begin{Shaded}
\begin{Highlighting}[]
\BuiltInTok{help}\NormalTok{(example\_function)}
\end{Highlighting}
\end{Shaded}

\begin{verbatim}
Help on function wrapper in module __main__:

wrapper(*args, **kwargs)
\end{verbatim}



\end{document}
